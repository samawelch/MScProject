%% Fair warning, gentle reader: at least half of the lines below are salty comments about having to use LaTeX.
%% What's worse, the other half are LaTeX.

%% This is file `elsarticle-template-1-num.tex',
%%
%% Copyright 2009 Elsevier Ltd
%%
%% This file is part of the 'Elsarticle Bundle'.
%% ---------------------------------------------
%%
%% It may be distributed under the conditions of the LaTeX Project Public
%% License, either version 1.2 of this license or (at your option) any
%% later version.  The latest version of this license is in
%%    http://www.latex-project.org/lppl.txt
%% and version 1.2 or later is part of all distributions of LaTeX
%% version 1999/12/01 or later.
%%
%% Template article for Elsevier's document class `elsarticle'
%% with numbered style bibliographic references
%%
%% $Id: elsarticle-template-1-num.tex 149 2009-10-08 05:01:15Z rishi $
%% $URL: http://lenova.river-valley.com/svn/elsbst/trunk/elsarticle-template-1-num.tex $
%%

\documentclass[final,1p,times]{elsarticle}

%% Use the option review to obtain double line spacing
%% \documentclass[preprint,review,12pt]{elsarticle}

%% Use the options 1p,twocolumn; 3p; 3p,twocolumn; 5p; or 5p,twocolumn
%% for a journal layout:
%% \documentclass[final,1p,times]{elsarticle}
%% \documentclass[final,1p,times,twocolumn]{elsarticle}
%% \documentclass[final,3p,times]{elsarticle}
%% \documentclass[final,3p,times,twocolumn]{elsarticle}
%% \documentclass[final,5p,times]{elsarticle}
%% \documentclass[final,5p,times,twocolumn]{elsarticle}

%% The graphicx package provides the includegraphics command.
\usepackage{graphicx}
%% The amssymb package provides various useful mathematical symbols
\usepackage{amssymb}
%% The amsthm package provides extended theorem environments
%% \usepackage{amsthm}
%% I cannot believe how little functionality is included natively in \LaTeX
\usepackage{rotating}
%% Some packages for less awful tables
\usepackage{booktabs}
\usepackage{tabularx}
\usepackage{array}
%% It's designed for scientific reports and yet crashes if you use SI!
\usepackage{siunitx}
%% It doesn't support text subscripts! It doesn't even document well that it doesn't support text subscripts!
\usepackage{tikz}
%% It doesn't support degree signs
\usepackage{textcomp}
\usepackage{pdflscape}
\usepackage{colortbl} 
\usepackage{xcolor} 
\usepackage{xfrac}
\usepackage[title]{appendix}
\usepackage{hyperref}
\hypersetup{final}
\usepackage{threeparttable}
\usepackage{scalefnt}
\usepackage[utf8]{inputenc}
\usepackage[T1]{fontenc}
\usepackage{amssymb}

%% Keeps tables looking tolerable.
\newcolumntype{L}{>{\centering\arraybackslash}m{2.7cm}}
\newcolumntype{M}{>{\centering\arraybackslash}m{3cm}}

%% Seeing as this is going to be submitted electronically, I think we can sacrifice page count at the altar of readability. 
\setlength{\parskip}{1em}

%% The lineno packages adds line numbers. Start line numbering with
%% \begin{linenumbers}, end it with \end{linenumbers}. Or switch it on
%% for the whole article with \linenumbers after \end{frontmatter}.
\usepackage{lineno}

%% natbib.sty is loaded by default. However, natbib options can be
%% provided with \biboptions{...} command. Following options are
%% valid:

%%   round  -  round parentheses are used (default)
%%   square -  square brackets are used   [option]
%%   curly  -  curly braces are used      {option}
%%   angle  -  angle brackets are used    <option>
%%   semicolon  -  multiple citations separated by semi-colon
%%   colon  - same as semicolon, an earlier confusion
%%   comma  -  separated by comma
%%   numbers-  selects numerical citations
%%   super  -  numerical citations as superscripts
%%   sort   -  sorts multiple citations according to order in ref. list
%%   sort&compress   -  like sort, but also compresses numerical citations
%%   compress - compresses without sorting
%%
\biboptions{comma,square}

% \biboptions{}

\journal{Science of the Total Environment}

\begin{document}

\begin{frontmatter}

%% Title, authors and addresses

\title{Are regulatory limits on pollutants sufficient to protect from emergent mixture effects?}

%% use the tnoteref command within \title for footnotes;
%% use the tnotetext command for the associated footnote;
%% use the fnref command within \author or \address for footnotes;
%% use the fntext command for the associated footnote;
%% use the corref command within \author for corresponding author footnotes;
%% use the cortext command for the associated footnote;
%% use the ead command for the email address,
%% and the form \ead[url] for the home page:
%%
%% \title{Title\tnoteref{label1}}
%% \tnotetext[label1]{}
%% \author{Name\corref{cor1}\fnref{label2}}
%% \ead{email address}
%% \ead[url]{home page}
%% \fntext[label2]{}
%% \cortext[cor1]{}
%% \address{Address\fnref{label3}}
%% \fntext[label3]{}


%% use optional labels to link authors explicitly to addresses:
%% \author[label1,label2]{<author name>}
%% \address[label1]{<address>}
%% \address[label2]{<address>}

\author{Samuel A. Welch}

\address{Silwood Park, Imperial College London, United Kingdom}

\begin{abstract}
%% Text of abstract
I did horrible things to bacteria; sometimes several at once. In the end though I think the bacteria may have still come out ahead. Maybe if I pad the abstract out a little more, like so, everything will be much more neatly laid out across the pages. 
\end{abstract}

\begin{keyword}
Bacteria \sep Ecotoxicology \sep Higher-order effects \sep Multiple Stressors \sep Mixture Toxicity

%% MSC codes here, in the form: \MSC code \sep code
%% or \MSC[2008] code \sep code (2000 is the default)

\end{keyword}

\end{frontmatter}

%%
%% Start line numbering here if you want
%%
\linenumbers

%% main text
\section{Introduction}
\label{S:1}
\subsection{The State of the Art}

Any organism today must content with not only a broad range of ecological and biogeographic stressors, but also the presence of more than 100 million known chemicals \cite{CAS2015}. Though many of these chemicals have always been naturally present in the world as components of the crust, secreted biocides or waste products, the chemical landscape of ecosystems has never been more complex, with chemical stressors mixing with hard-to-predict effects \cite{EuropeanCommission2012a}. 

Though mixture toxicity has long attracted much academic attention and study \cite{Bliss1939}, criticisms remain of a lack of consensus on experimental design, use of statistics, and even inconsistent interaction definitions \cite{Jackson2016,Piggott2015,Schafer2018}. Ecotoxicology still bears the legacy of its largely regulatory past, a preoccupation with declaring any given chemical safe or not, using a limited panel of model species \cite{OECD2014SectionSystems}. Equally, existing mixture toxicity regulations in the EU are limited and distinctly human-focused \cite{EuropeanCommission2012a}, and do not represent the latest in scientific knowledge.

A longstanding criticism of ecotoxicology remains the difficulty of integrating ecotoxicological results into a broader understanding of how ecosystems respond to stress across their component species. \cite{Chapman2002a,Gessner2016}. Ecosystems are without exception supported by a broad trophic foundation of microbes that provide vital services as decomposers, chemical engineers, and pathogens \cite{Nannipieri2003a,VanderHeijden2008a}, but in regulatory and scientific settings a chemical's toxicity to bacteria is typically tested using the luminescent bacteria toxicity assay (LBTA) \cite{OECD2014SectionSystems}, a 30-minute exposure to \textit{Aliivibrio fischerii}, a bioluminescent marine decomposer. The LBTA is the industry standard due to its simplicity and ease of use, but it suffers from a number of well-documented limitations \cite{Ma2014} in addition to the difficult of generalising from one species to an entire ecosystem.

Traditional models of mixture toxicity focus on binary mixtures of stressors. Two chemicals or environmental factors, that known to have a positive or negative effect on some aspect of a species' fitness, are mixed at various concentrations. Their combined effect compared to a null-model prediction of their interaction -- often additive, the sum of their parts -- in order to determine if in combination they display synergism -– an effect greater than the sum of their parts, antagonism – an effect lesser than the sum of their parts –- or simple additivism, when the effect is statistically indistinguishable from the null model. 

Though easy to understand and employ, this model is not without its issues. What, for example, is a stressor? The term is often used as a synonym for 'pollutant', 'toxicant', or 'pressure' -- a substance or treatment that produces a negative fitness effect in the species of interest \cite{Piggott2015}. However, not only does this fail to account for the well-known adage that 'the dose makes the poison'\cite{OGParacelsus} –- as true for a rise in temperature or in nutrient availability as it is for toxic pollutants -- but also that \textit{the species makes the poison}. Many organisms, particularly bacteria, are are able to tolerate or even thrive on substances that are harmful to other species \cite{Malik2004,Gadd2009}. 

Due to the geometrically-increasing complexity of adding additional stressors to interaction experiments, the number of studies that have experimentally examined the interactions of mixtures of three or more chemicals –- 'higher order interactions' –- has been limited \cite{Beppler2016UncoveringStressors}. This has been justified by a long-held assumption that higher-order interactions are negligible compared to lower-order interactions \cite{Tekin2017PrevalenceInteractions}, but recent studies examining complex mixtures of antibiotics have shown emergent higher-order effects,  three-way interactions not predictable from component two-way effects, meaning that the existing body of binary-interaction based literature may not be scalable over more complex mixtures to understand the effects on ecosystems \cite{Tekin2016, Zimmer2016, Tekin2017}.

Within ecology, research on complex combinations of stressors -– including chemicals –- is more common, and suffers from similar uncertainty and confusion in defining synergy and antagonism \cite{Cote2016}.  Beppler \textit{et al.} also highlight the study of emergent interactions between higher-order mixtures of predators -– Multiple Predator Effects (MPEs) –- that has received some attention. One recent review of the literature \cite{Griffin2013EffectsMeta-analysis} found overall that species-rich mixtures of predators were more effective at suppressing prey than poorer mixtures, although a considerable number of component studies found no significant relationship. 

In the Anthropocene, an era where pollution is a greater pressure on ecosystems than ever before, it is vital that we improve our understanding of how a diverse range of wild bacteria respond to different stressors. Greater understanding of these stressors and interactions will enable more targeted and cost-efficient legislation and interventions -– for instance, one stressor has a disproportionate effect on overall response and thus should produce the greatest positive effect when removed. 

This study thus aims to address this knowledge gap through the use of automated experimental set-up and observational equipment to examine the effects of complex mixtures of common pollutants on a diverse panel of soil bacteria. It is hypothesised that more complex mixtures of chemicals will produce emergent effects not predictable from their component lower-order interactions, and that single stressor responses and interaction responses will vary across the bacterial panel.

\section{Methods}
\label{S:2}
\subsection{Isolate Selection}
\label{S:2:1}

8 sets of bacterial isolates  were selected, sourced principally from the Nash's Field experimental site at Silwood Park, United Kingdom. Plots N and P (see Appendix \ref{A:2}) of Nash's Field have since 1991 been treated three times yearly with metaldehyde at 960 g/ha as part of herbivore exclusion experiments \cite{Allan2011ContrastingExperiment}, while bacteria were isolated and sequenced in 2016 for an earlier study \cite{Mombrikotb2016}. 

\begin{table}[ht]
\begin{small}
\centering
\begin{tabular}{c p{3.1cm} l p{5.2cm}}
\toprule 
{Strain} & {Species} & {History} & {Notes} \\
\midrule
\rowcolor{black!20}{LUF4\_5} & \textit{Luteibacter rhizovicinus} & Control & First isolated from the rhizosphere \cite{Johansen2005LuteibacterL.}. \\
{KUB5\_13} & \textit{Variovorax paradoxus} & Control & Strains capable of metabolising a wide range of pollutants \cite{Satola2013MetabolicParadoxus}. \\
\rowcolor{black!20}{NUF1\_3} & \textit{Variovorax paradoxus} & Metaldehyde & \\
{KUE4\_4} & \textit{Bacillus simplex} & Control & Strains capable of biosorption of heavy metals and radionuclides \cite{Valentine1996BiosorptionZone}. \\
\rowcolor{black!20}{NUE1\_1} & \textit{Bacillus simplex} & Metaldehyde & \\
{KUE4\_10} & \textit{Stenotrophomonas \newline acidaminiphila} & Control & First isolated from a waste-acid-treating anaerobic bioreactor \cite{Assih2002}. \\
\rowcolor{black!20}{OP50} & \textit{Escherichia coli} & Control & Included as a model species. \\
{KUA5 Sample} & Soil Community & Control & \\
\bottomrule
\end{tabular}
\caption{Bacterial species, strains, and sources.}
\label{tab:isolates}
\end{small}
\end{table}

Isolates were stored on CryoBeads in a standard \textit{Brucella} broth with glycerol solution (Hardy Diagnostics, CB12) at -80\textdegree C until needed for experiments, when they were cultured overnight on a rotary shaker in a standard R2A broth at room temperature (25.9\textdegree C–-26.6\textdegree C over a 48h period) prior to experiments. Soil communities were frozen in a 60\% glycerol solution immediately after extraction and defrosted on an individual basis for overnight culturing, meaning it was probably that each replicate of soil community exposures contained a distinct diversity and abundance of species in itself. 

\subsection{Stressor Selection}
\label{S:2:2}

Stressors were selected across a diverse range of functional groups (Table \ref{tab:stressors}) in order to assemble a panel that had some but not all mechanisms of action in common amongst the stressors.

Nickel (\textsubscript{28}Ni) and Copper (\textsubscript{29}Cu) are heavy metals, common pollutants with a wide range of industrial applications. Copper is an essential respiratory nutrient across all kingdoms of life \cite{Babcock1992OxygenRespiration}, while nickel is occasionally so in bacteria and fungi \cite{Zamble2015NickelBiology}. Bacterial resistance to these stressors is thus often nuanced, with a requirement to balance availability in the cell as nutrients with their potential for damage.  Copper is a prolific generator of Reactive Oxygen Species (ROS) \cite{Bal2002InductionMetals}, and damages vital biosynthesis enzymes \cite{Macomber2009TheToxicity}, while nickel is a weak ROS generator that can unbalance iron and zinc homeostasis in the cell \cite{Samland2006MicrobialDevelopments}. Copper and nickel are resisted through similar pathways \cite{Mykytczuk2011CytoplasmicFerrooxidans}, including active efflux and membrane modification. Copper is also resisted through chelation and rapid repair of damaged enzymes \cite{Macomber2009TheToxicity}, while nickel can be sequestered inside the cell \cite{Nishimura1998ProtonCerevisiae}.

Chloramphenicol and ampicillin are broad spectrum antibacterial agents, used in decreasing amounts in healthcare applications due to their severe side effects and growing resistance, but nevertheless well-studied environmental pollutants. First isolated from \textit{Streptomyces venezuelae}, a soil-dwelling bacterium, chloramphenicol is a broad-spectrum antibiotic to which resistance in the wild and areas under antibiotic pollution is particularly common \cite{Allen2010CallEnvironments}. Ampicillin, a widespread, broad-spectrum antibiotic from the penicillin family is also widely resisted \cite{Ruiz1999MechanismsFish}. Chloramphenicol acts bacteriostatically by inhibiting protein synthesis in the 50S ribosomal subunit, while ampicillin inhibits cell wall synthesis. Drug interaction studies have shown that chloramphenicol is negatively antagonistic towards ampicillin due to their competing modes of action \cite{vanBambeke2017MechanismsAction}. It has been suggested that many of the genes that provide resistance to common antibiotics including chloramphenicol also provide tolerance to environmental stress in non-pathogenic species \cite{Groh2007GenesResistance}. Research has also found that exposure to ROS from heavy metals, including copper, can co-select for chloramphenicol resistance \cite{Harrison2009ChromosomalTolerance}.

Metaldehyde and atrazine are two distinct chemical pesticides; the first a molluscicide, the second a herbicide. Atrazine, a triazine pesticide, has been banned in the EU since 2004 \cite{EU2004CommissionSubstance}, but has remained the most commonly used herbicide in the US, and acts on plants by disrupting photosynthesis \cite{Shimabukuro1969AtrazineAction}, while metaldehyde is rapidly converted within the body of molluscs to aldehyde, which damages mucus producing cells, causing excessive mucus production, dehydration, and eventual death \cite{Triebskorn1998}. Atrazine has been shown to be both a food source \cite{Wackett2002BiodegradationStudies} and ROS stressor \cite{Zhang2012OxidativeAtrazine} to various species of bacteria, but information on metaldehyde’s effects on bacteria are limited: one study has examined interactions between bacteria and metaldehyde \cite{Thomas2017IsolationSoils}, showing only that \textit{Variovorax} and \textit{Aceinetobacter} strains can be isolated from metaldehyde-treated soil and can degrade the molluscicide.

Tebuconazole is a triazole fungicide that acts against a broad range of pathogens by inhibiting fungi-specific membrane synthesis pathways. Tebuconazole is known to be toxic to a range of non-target species \cite{Sehnem2010}, but information on effects on bacteria is limited to the knowledge that some bacteria (not including any of the species used in this study) are capable of degrading tebuconazole \cite{Sehnem2010}. Azoxystrobin is a systemic fungicide in heavy use due to its broad-spectrum inhibition of respiration across major groups of fungal pathogens. Azoxystrobin has been shown to inhibit bacterial growth in mixed fungal-bacteria communities \cite{Bacmaga2015MicrobialAzoxystrobin}, although the same study showed \textit{Bacillus} species were capable of growth in highly contaminated soil.

%% LaTeX was sent to punish us for our sins.
%% I've got no idea what they were, but they must have been pretty bad for us to end up with LaTeX.

\begin{landscape}
\begin{table}[ht]
\begin{threeparttable}
\small
\setlength{\tabcolsep}{4pt}
\setlength{\extrarowheight}{3pt}
\begin{tabular}{l M l M L L c}
\toprule
\textbf{Stressor} & \textbf{Functional Group} & \textbf{Limit (\SI{}{\ug}/l}) & \textbf{Form/Product Code} & \textbf{Bacterial mechanisms of action} & \textbf{Bacterial mechanisms of resistance} & \textbf{Sources} \\
\midrule
\rowcolor{black!20}\textbf{Copper} & Heavy Metal & 2000\tnote{t} & Cu(II)Cl\textsubscript{2}, dihydrate (99\%), Alfa Aesar 12458 & Essential nutrient, ROS, enzymes & Efflux, chelation, rapid repair, membrane transition & \cite{Nayar2004EnvironmentalMesocosms,Valko2005MetalsStress,Dupont2011CopperApplications,ECHA2018REACHCopper} \\
\textbf{Nickel} & Heavy Metal & 20\tnote{t} & Ni(II)Cl\textsubscript{2}, anhydrous (98\%), Alfa Aesar B22085 & Enzymes, Fe/Zn homeostasis, weak ROS, nutrient & Sequestration, efflux, membrane transition & \cite{Nayar2004EnvironmentalMesocosms,Macomber2011,Nishimura1998ProtonCerevisiae,Zamble2015NickelBiology} \\
\rowcolor{black!20}\textbf{Chloramphenicol} & Antibacterial & 0.05\tnote{n} & Powder $\geqslant$98\%, Sigma Aldrich C0378 & Cell wall synthesis & Membrane transition, mutant ribosomes, anti-AB enzymes & \cite{Shaw1979Primary31,Rebstock1949ChloramphenicolChloromycetin,Toku-E2018ChloramphenicolChloromycetin,Ruiz1999MechanismsFish} \\
\textbf{Ampicillin} & Antibacterial & 0.12\tnote{n} & Ampicillin sodium salt, Sigma Aldrich A9518 & Protein synthesis, 50S ribosome subunit & $\beta$-lactamase enzymes, efflux & \cite{Ruiz1999MechanismsFish,Costanzo2005EcosystemEnvironment} \\
\rowcolor{black!20}\textbf{Atrazine} & Pesticide (Herbicide) & 0.25\tnote{t} & Power, analytical, Sigma Aldrich 45330 & Oxidative stress, bacterial enzymatic nutrient & Biodegradation & \cite{Shimabukuro1969AtrazineAction,Delorenzo2001TOXICITYREVIEW,Zhang2012OxidativeAtrazine} \\
\textbf{Metaldehyde} & Pesticide \break (Insecticide) & 0.5\tnote{t} & Powder, analytical, Sigma Aldrich 63990 & Toxicity unknown, possible nutrient & Biodegradation & \cite{Kay2014UsingProblem,Castle2017,Thomas2017IsolationSoils} \\
\rowcolor{black!20}\textbf{Tebuconazole} & Antibacterial & 1\tnote{r} & Ampicillin sodium salt, Sigma Aldrich 32013 & Unknown & Biodegradation & \cite{Sehnem2010,Artigas2014ComparativeEcosystems} \\
\textbf{Azoxystrobin} & Antibacterial & 3\tnote{r} & Powder, Analytical, Sigma Aldrich 31697 & Unknown & Biodegradation & \cite{Battaglin2011Occurrence20052006,Rodrigues2013,Loos2010Pan-EuropeanWater,Bacmaga2015MicrobialAzoxystrobin} \\
\bottomrule
\end{tabular}
\caption{Summary of stressors including type, target concentration, product information and bacterial interactions.}
\label{tab:stressors}
\begin{tablenotes}
\item [t] UK drinking water regulatory limit, Water Framework Directive
\item [n] No legal limit, typical environmental concentrations used.
\item [r] Regulatory Acceptable Concentrations, Water Framework Directive 
\end{tablenotes}
\end{threeparttable}
\end{table}
\end{landscape}

%% There! A table that look five times as long to produce as in Word, and is twice as ugly.

\subsection{Rangefinding and Concentration Calculations}
\label{S:2:3}

Initial dose-response exposures were conducted at concentrations above and below regulatory limits to obtain an overview of individual stressors' effects on isolates. Eight 96-well microtiter plates were prepared, with  \SI{10}{\ul} of 1-in-100 diluted overnight culture, \SI{80}{\ul} of R2A broth, and \SI{10}{\ul} of stressor stock at either 0.1, 1, 10 or 100 times the target final experimental concentration aliquoted into each well. Each well of stressor concentration and isolate was replicated three times. Well OD was read using the cell count protocol below  \ref{S:2:4}.

Final experimental concentrations were calculated based on well volumes and regulatory limits. Calculations are available in Appendix \ref{A:1}.

\subsection{Metrics of Growth}
\label{S:2:4}

Microcosm optical density at 590 nm was used as a metric of cell count over time. Immediately after isolates were exposed to stressors, plates were placed in an automatically-fed BioTek Synergy 2 microplate reader for 48 hours, agitating the wells for 5 seconds then reading absorbance every four hours.  At the conclusion of OD readings, wells were checked for contamination by plating the contents of the control plate at $10^{-3}$ and $10^{-4}$ dilutions onto a R2A agar plate and culturing at room temperature for 24 hours.

Optical density was modelled against time in hours using a logistic curve from the package growthcurver \cite{Sprouffske2016Package} for the R programming language \cite{RCoreTeam2018R:Computing}

\begin{equation}
Nl_t=\frac{K}{1 + (\frac{K - N_0}{N_0}) e^{-rt}}
\label{E:log_curve}
\end{equation}

Growth data and logistic curve were plotted by well, allowing for visual inspection and the exclusion of models from wells where no growth had occurred as a result of factors other than stressor treatment. A null additive model was assumed for all comparisons, consistent with the use of bacterial growth as a metric of response \cite{Piggott2015}. Empirical area under the measured growth (\textit{auc\_e} in growthcurver, henceforth growth) was used as a primary measure of bacterial growth response, as it allowed for taking account of effects on growth at any period in the experiments.

\subsection{Stressor Exposures}
\label{S:2:5}

Stressor combinations were formulated by use of a Hamilton MicroLab STARLet, fitted with sterile  pipette tips and a laminar flow hood. 255 combinations of 8 stressors across 8 bacterial isolates were formulated across 24 \SI{2}{\ul} flat bottom 96-well plates, with additional negative controls for a total of 2144 exposures per replicate. \SI{10}{\ul} doses of stressor solutions at environmentally relevant concentrations (Table \ref{tab:stressors}) were added to the wells, in addition to \SI{10}{\ul} of overnight bacterial culture diluted to 1 in 1000, and sufficient R2A broth to bring all well volumes up to \SI{100}{\ul}. Machine-readable worklists were generated from a combination input file for the STARLet using an R script (Appendix \ref{A:1}). Experimental for a single replicates typically lasted 20 hours spread across 2 days; plates were stored in sealed containers at 4\textdegree C overnight, and from the second run onwards with a dampened paper towel to reduce evaporation.

Due to a number of issues with isolate growth in liquid media and plate inoculation, sample size varies across both treatments and isolates (Table \ref{tab:samples}). 

\begin{table}[ht]
\centering
\small
\begin{tabular}{l c c}
\toprule 
\textbf{Isolate} & \textbf{n Mean} & \textbf{n SD}  \\
\midrule
\rowcolor{black!20}{LUF4\_5} & 3.3 & 1.7 \\
{KUB5\_13} & 2.6 & 1.9 \\
\rowcolor{black!20}{NUF1\_3} & 2.7 & 1.7\\
{KUE4\_4} & 3.4 & 1.8\\
\rowcolor{black!20}{NUE1\_1} & 2.6 & 1.6\\
{OP50} & 3.3 & 2.0 \\
\rowcolor{black!20}{Soil Community} & 3.0 & 1.7\\
\bottomrule
\end{tabular}
\caption{Sample size mean and standard distribution by isolate.}
\label{tab:samples}
\end{table}

\subsection{Statistical Analysis}
\label{S:2:6}
All data processing and statistical analysis was conducted in R version 3.5.1 \cite{RCoreTeam2018R:Computing}, data using the \textit{tidyverse} family of packages \cite{Wickham2017Tidyverse:Tidyverse}. All R code is available in Appendix \ref{A:1}, as well as full attribution of all other packages, functions and software used in the production of this report.

\subsection{Mixture Complexity Versus Growth}
\label{S:2:7}
Growth was modelled against the number of stressors applied in a treatment ---"stressor complexity"--- using two linear models with complexity and isolate or species as explanatory variables, which were compared using an analysis of variance (ANOVA). A mean across all isolates was also calculated. 

\subsection{Single Stressor Growth Curves by Isolate}
\label{S:2:8}
In order to determine effect direction, size and significance at the individual stressor level, growth curves were fitted as above for individual combinations of stressors and isolates. 

\subsection{Interaction Type Prevalence}
\label{S:2:9}
Stressor effect was calculated by subtracting the average growth with stressors from the control growth for a given isolate, producing an additive measure of effect, which was averaged across replicates to produce a measure of mean growth and standard deviation.
\begin{equation}
Effect = Growth_{Control} - Growth_{Treatment}
\label{E:effect_calc}
\end{equation}

Predicted effect was then calculated using an additive null model, by summing the calculated mean effects of individual stressors on a per-isolate basis and calculating an updated standard deviation from the root of the summed squares of the component standard deviations. 

Effect type was classified according to the definitions found in Piggott, Townsend, and Matthaei \cite{Piggott2015}. If the mean predicted and observed effects were not significantly different (independent two-sample T-test, p $<$ 0.05). A positive effect on growth larger than the predicted effect was classified as positive synergy, a negative effect larger (in absolute terms) than the predicted effect was negative synergy. An effect between positive synergy and the predicted effect was classified as negative antagonism, while an effect between negative synergy and the predicted effect was positive antagonism. 

\subsection{Higher Order Mixtures and Emergent Interactions}
\label{S:2:10}
Eight multiple linear regressions were fitted to the data using 1-8-way combinations of stressors to explain variation, and determine statistically what level of interaction was predominant in the observed results. Linear regressions were compared using an ANOVA and Akaike Information Criteria test (AIC). 

A variation of the protocol used in Beppler \textit{et al.} \cite{Beppler2016UncoveringStressors} was applied to the data to detect 'emergent interactions', observed effects for an c-complexity stressor mixture that could not be predicted from the combined effects of their component ($c-1$)-complexity mixture observations. Predicted mean, standard deviation and sample size were calculated per isolate as follows: 
\begin{equation}
    m_c = \sum_{i = 1}^{c_n}\Big(\frac{m_c-1}{c_n}\Big)
\label{E:pred_mean}    
\end{equation}
\begin{equation}
    s^2_n = \sqrt{\sum_{i = 1}^{c_n}\bigg(\frac{(s^2_{n-1})^2}{c_n}\bigg)}
\label{E:pred_sd}    
\end{equation}
\begin{equation}
    n_c = \sum_{i = 1}^{c_n}\Big(\frac{n_c-1}{c_n}\Big)
\label{E:pred_n}    
\end{equation}
where:

\begin{small}
\setlength{\tabcolsep}{1pt}
\begin{tabular}{r p{11cm}}
    $c :$ & complexity of stressor mixture \\
    $c_n :$ & number of mixtures of $c$ complexity \\
    $m :$ & mean effect of mixture of complexity $c$ \\
    $n_c :$ & adjusted sample size of component--predicted effect of mixture of complexity $c$ \\
    $s^2_n :$ & adjusted summed standard deviation of component--predicted mixture of complexity $c$ \\
\end{tabular}
\end{small}

\section{Results}
\label{S:3}

\subsection{Rangefinding}
\label{S:3:1}

Isolates showed a variable response across the spread of rangefinding concentrations with KUB5\_13, KUE4\_4 and the Nash's Field Soil Community generally proving the most able to grow regardless of stressor or concentration. Mean response to concentration (in cyan) showed a variety of responses across the stressors, although the expected negative correlation between concentration and mean growth was not apparent. Azoxystrobin--exposed cultures did not appear to grow at any of the concentrations used.

\subsection{Growth by Stressor Complexity}
\label{S:3:2}

Modelling mean growth as either isolate or species versus mixture complexity did not reveal a statistically significant relationship between complexity and growth. Species identities \textit{B. simplex}, \textit{E. coli}, \textit{V. paradoxus} and the soil community were good predictors of variation (p $<$ 0.05).

\subsection{Single Stressor Responses}
\label{S:3:3}

Isolates displayed a variety of responses to the various stressors, growth in some cases apparently enhanced by the presence of most stressors (NUE1\_1, KUE4\_10, OP50, KUE4\_4, and the soil community) but in some cases diminished (NUF1\_3 and LUF4\_5) and in one largely unaffected (KUE4\_4). A bump in  the growth phase generally indicative of diauxic growth was strongly apparent in KUE4\_10, and sporadically present in the soil community, KUE4\_4, NUF1\_3 and NUE1\_1. 

\subsection{Predicted Versus Observed Effect}
\label{S:3:4}

Patterns of observed versus predicted mean effect varied considerably between the species. KUB5\_13, NUF1\_3 and LUF4\_5’s are tight, linear distributions clustered around the 0 marks on both axes, while OP50, KUE4\_4 and the soil community’s observed effects fell largely below the predicted additive effect, especially at high levels of stressor complexity. KUE4\_10 and NUE1\_1’s patterns are more diffused across the additive line, but display a similar pattern of synergy at lower levels of complexity and antagonism at higher levels. This pattern is repeat across the majority of isolates, although LUF4\_5 shows a reversal of this pattern, and NUF1\_3 is clustered similarly but with greater x and y overlap between points of different richness.

\subsection{Interaction Type Prevalence}
\label{S:3:5}

The analysis of interaction type showed a relatively high incidence of T-test errors (133 in 1976; ~15\%), due to a correspondingly high incidence of observations with a sample size of one and no standard deviation. For most isolates additive effects prevailed (1744 in 1976; ~88\%), the exception being \textit{E. coli} OP50, where some non-additive effects were seen, especially at a complexity of 4 and above; negative antagonism accounted for slightly under half of the interactions observed (94 in 247; 38\%), while negative synergy was observed only at a complexity of 5 and in low quantities (5 in 247; 2\%).

\subsection{Emergent Interaction Type Prevalence}
\label{S:3:6}

T-test errors represent a similar proportion of emergent effects observed (161 in 1752, ~9\%) . Non-emergent effects – “predicted” – dominate (1589 in 1752, ~91\%). Emergent synergy is non-existent, while two cases of emergent antagonism are present in NUE1\_1: a mixture of copper, nickel, metaldehyde and tebuconazole, and a mixture of copper, nickel, chloramphenicol, metaldehyde, tebuconazole, and azoxystrobin.

\section{Discussion}
\label{S:4}

\subsection{Novel Findings}
\label{S:4:1}

Of the two hypotheses laid down in this project’s introduction, limited conclusions can be drawn. Figure 2 shows various positive and negative relationships between mixture complexity and growth across the species, although none of these relationships are statistically significant. 

Although Figure 4 shows a relatively high incidence of observed effects deviating from the additive prediction, very few of these deviations are significant at p $<$ 0.05 (Figure 5), the few cases of synergy and antagonism being limited to \textit{E. coli} OP50, where they were relatively common. A similar lack of emergent interactions (Figure 6) was detected. 

However, a degree of interspecies variation at various levels of significance in response and interactive effect was present. The effects of individual stressors on growth varied considerably and frequently reversed between species, and different patterns of clustering between observed and predicted effects were observed in Figure 4. As discussed above, \textit{E. coli} OP50 showed the only significant non-additive interactions detected, although no similar pattern was observed at the emergent level. 

\subsection{Single Stressor Effects}
\label{S:4:2}

Although the majority of the stressors used in this study have well-documents impacts on bacteria, a number – including the pesticides atrazine and metaldehyde and the fungicides tebuconazole and azoxystrobin – have little to no literature available on their effects or mechanism of effect on non-target species and indeed may represent a food source of a stressor for different species under different conditions, which appears to have been the case in single stressor exposures.

\subsection{Caveats and Limitations}
\label{S:4:3}

A number of limitations apply to this study if overcome could affect the significance of detected interactions considerably.  Due to a combination of mistakes during the lab work phase of the study and the reticence of the strains KUB5\_13 and NUF1\_3 (\textit{V. paradoxus}) to grow directly in liquid media from cryogenic storage (perhaps unsurprising as the strains in question were sources from soil samples), the sample size in the case of a number of combinations (Table \ref{tab:samples}) was insufficiently large to detect non-additive effects at all but the highest p-values.  Greater replication of a smaller number of isolates would have permitted results to be declared with a greater degree of certainty. 

Another issue that may have affected OD readings for strains KUE4\_4 and NUE1\_1 (\textit{B. simplex}) is the tendency of Bacillus to aggregate within the well, creating considerable local variation in optical density across 
Additionally, due to the long preparation times for exposures, and experimental durations (96 hours from first the addition of media to plates to experiment end), well evaporation posed a non-random but difficult to account for influence on bacterial growth conditions, an additional stressor not part of the experimental design. Evaporation was most evident in peripheral wells (A1-–12, H1-–12, all X1 and X12 wells), and as stressor combination layout on the plates was not randomised between isolates or runs, it is probably that evaporation played a not-insignificant role in the observed effects of various treatments. Were a simpler level of treatment/isolate complexity used in simpler work, it would be possible to mitigate this effect by using only the inner wells of the 96 well plate, and by reduce experimental set-up time. 

Lastly, the incorporation of varying concentrations of stressors would have allowed the construction of dose response curves, calculation of inhibitory concentrations (ICs), and correspondingly the modelling of Bliss Independence \cite{Bliss1939} of stressors, which has been applied with some success to the issue of higher-order and emergent interactions in complex mixtures of antibiotics \cite{Beppler2016UncoveringStressors,Tekin2017PrevalenceInteractions} 
\subsection{Next Steps}
\label{S:4:4}

A small but growing body of evidence conducted by researchers at the University of California has suggested that, in lab strains of \textit{E. coli}, the potential for higher-order interactions in complex mixtures of chemicals is far higher than previously assumed \cite{Beppler2016UncoveringStressors,Tekin2016,Tekin2017PrevalenceInteractions}.  This study has undertaken a similar approach to a more diverse mixture of chemicals with mixed results; however, it is clear that existing attempts to understand the effects of the complex mixtures of stressors that define interactions in real-world ecosystems through simple interactions is insufficient.

With the increasing level of automation available in laboratories, it is crucial that further steps are taken to undertake work that will improve our understanding the role of complex stressor mixtures in contemporary ecosystems and study systems, and to develop effective strategies to protect and restore such effected ecosystems.

\section{Conclusions}
\label{S:5}

Recent advances in the understanding of higher-order effects of complex stressor mixtures suggest that effects are not as previous believed negligible. This study found predominantly additive effects at both the interaction and emergent interaction level; with non-additive effects largely limited to negative antagonism for \textit{E. coli} OP50 exposed to more complex mixtures. This suggests that studies that examine only lab species for interactions may not be fully applicable to wild populations, and further study is needed to determine applicability to the diversity of species present in the wild.

\section{Acknowledgements}
\label{S:6}
%First and foremost I would like to thank my parents, without whose patience, support, and advice none of which would be possible.

%Over the course of this project I benefited from the guiding influence of my official supervisors, Dr Tom Bell and Dr Emma Ransome, who were encouraging, supportive, educational, and, when necessary, firm. Dr Shorok Mombriktob was also of indispensable assistance, providing isolates, soil samples, advice, and training on use of certain temperamental pieces of lab equipment. Chris,  Nilita, Kate, Alice, and Meirion, also of Unit C, helped with other elements of my labwork.

%Lastly in addition to the academic and commercial software and software packages cited in the main text and reference, I would like to thank to thank Chris West, who produced a batch Excel worksheet to .csv converter which was invaluable in the processing of my raw data. I've since learned that I had an R package that does the same thing installed the whole time...

\section{References}
\label{S:7}
%% References
%%
%% Following citation commands can be used in the body text:
%% Usage of \cite is as follows:
%%   \cite{key}          ==>>  [#]
%%   \cite[chap. 2]{key} ==>>  [#, chap. 2]
%%   \citet{key}         ==>>  Author [#]
%% References with bibTeX database:
\bibliographystyle{model1-num-names}
\bibliography{mendeley.bib}
%\end{thebibliography}
\begin{appendices}

\section{Data and Code}
\label{A:1}

All data, scripts, and calculations used in this study are available at: 

\url{https://github.com/samawelch/MScProject}

\section{Nash's Field Treatments and Isolates}
\label{A:2}


\section{Notes on the Operation of Hamilton STARlet}
\label{A:3}
\end{appendices}

\end{document}